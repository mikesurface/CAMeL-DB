\begin{abstract}
\sean{Temporary, will be rewritten.}
The need to automatically process and store large amounts of uncertain and imprecise machine learned data has necessitated the use of Probabilistic Databases (PDBs) which maintain and allow queries that carry a degree of uncertainty.  An integral part of the data cleaning process is finding efficient ways to reduce this uncertainty.  In this paper we propose the Crowd Assisted Machine Learning (CAMeL) paradigm which seeks to utilize crowdsourcing techniques such as Amazon Mechanical Turk to optimally improve the accuracy of learned data.  This paradigm is implemented on top of a probabilistic database which we call CAMeL-DB.  A subset of the automatically populated fields are converted into questions to be answered by the crowd.  We demonstrate the efficacy of our approach on an information extraction problem consisting of automated segmentation of bibliographic citations, showing that a relatively small subset of questions can lead to a large boost in accuracy.
\end{abstract}
