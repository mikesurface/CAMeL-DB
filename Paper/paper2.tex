% This is "sig-alternate.tex" V2.0 May 2012
% This file should be compiled with V2.5 of "sig-alternate.cls" May 2012
%
% This example file demonstrates the use of the 'sig-alternate.cls'
% V2.5 LaTeX2e document class file. It is for those submitting
% articles to ACM Conference Proceedings WHO DO NOT WISH TO
% STRICTLY ADHERE TO THE SIGS (PUBS-BOARD-ENDORSED) STYLE.
% The 'sig-alternate.cls' file will produce a similar-looking,
% albeit, 'tighter' paper resulting in, invariably, fewer pages.
%
% ----------------------------------------------------------------------------------------------------------------
% This .tex file (and associated .cls V2.5) produces:
%       1) The Permission Statement
%       2) The Conference (location) Info information
%       3) The Copyright Line with ACM data
%       4) NO page numbers
%
% as against the acm_proc_article-sp.cls file which
% DOES NOT produce 1) thru' 3) above.
%
% Using 'sig-alternate.cls' you have control, however, from within
% the source .tex file, over both the CopyrightYear
% (defaulted to 200X) and the ACM Copyright Data
% (defaulted to X-XXXXX-XX-X/XX/XX).
% e.g.
% \CopyrightYear{2007} will cause 2007 to appear in the copyright line.
% \crdata{0-12345-67-8/90/12} will cause 0-12345-67-8/90/12 to appear in the copyright line.
%
% ---------------------------------------------------------------------------------------------------------------
% This .tex source is an example which *does* use
% the .bib file (from which the .bbl file % is produced).
% REMEMBER HOWEVER: After having produced the .bbl file,
% and prior to final submission, you *NEED* to 'insert'
% your .bbl file into your source .tex file so as to provide
% ONE 'self-contained' source file.
%
% ================= IF YOU HAVE QUESTIONS =======================
% Questions regarding the SIGS styles, SIGS policies and
% procedures, Conferences etc. should be sent to
% Adrienne Griscti (griscti@acm.org)
%
% Technical questions _only_ to
% Gerald Murray (murray@hq.acm.org)
% ===============================================================
%
% For tracking purposes - this is V2.0 - May 2012

\documentclass{vldb}

% comments
\newcommand{\mike}[1]{{\textcolor{green}{#1 -- MJF}}}
\newcommand{\joe}[1]{{\textcolor{red}{#1 -- JMH}}}
\newcommand{\minos}[1]{{\textcolor{cyan}{#1 -- MG}}}
\newcommand{\daisy}[1]{{\textcolor{magenta}{#1 -- DZW}}}
\newcommand{\sean}[1]{{\textcolor{blue}{#1--SLG}}}


% general
\renewcommand{\ttdefault}{cmtt}
\newcommand{\rbox}{\hfill $\Box$}

\newtheorem{thm}{Thorem}
\newtheorem{lem}{Lemma}
\newtheorem{ex}{Example}
\newtheorem{df}{Definition}[section]

\newcommand{\eat}[1]{}
\newcommand{\vpar}{\vspace*{.5em}}
\newcommand{\cut}[1]{}

\newenvironment{compactitemize}%
    {\begin{list}{}{
       \renewcommand{\makelabel}[1]{\bf $\bullet$}\hfil%
       \settowidth{\labelwidth}{\bf $\bullet$}%
       \setlength{\partopsep}{0mm}%
       \setlength{\parsep}{0mm}%
       \setlength{\parindent}{0mm}%
       \setlength{\itemsep}{0mm}%
       \setlength{\topsep}{0mm}%
       \setlength{\leftmargin}{\labelwidth}%
       \addtolength{\leftmargin}{\labelsep}
     }}%
    {\end{list}}

% BayesStore
\newcommand{\bs}{\textsc{BayesStore}\xspace}
\newcommand{\largebs}{{\large{\textsc{BayesStore}}}\xspace}
\newcommand{\Largebs}{{\Large{\textsc{BayesStore}}}\xspace}
\newcommand{\LARGEbs}{{\LARGE{\textsc{BayesStore}}}\xspace}
\newcommand{\hugebs}{{\huge{\textsc{BayesStore}}}\xspace}

% VLDB08 Abbreviations:
\newcommand{\ir}{$R$\xspace}
\newcommand{\sensor}{\textsl{Sensor1}\xspace}
\newcommand{\irset}{$\mathcal{R}$\xspace}
\newcommand{\pr}{\mbox{\sf Pr}}
\newcommand{\dom}{\mbox{\sf dom}}
\newcommand{\pdf}{$F$\xspace}
\newcommand{\true}{$true$\xspace}
\newcommand{\false}{$false$\xspace}
%\newcommand{\model}{$\mathcal{M}$}
\newcommand{\mv}{\textsl{miss}\xspace}
\newcommand{\RV}{\textsc{rv}\xspace}
\newcommand{\RVs}{\textsc{rv}s\xspace}
\newcommand{\bns}{\textsc{BN}s\xspace}
\newcommand{\mrfs}{\textsc{MRF}s\xspace}
\newcommand{\mrf}{\textsc{MRF}\xspace}
\newcommand{\rv}{$X$}
\newcommand{\pa}{$\mathcal{A}^p$\xspace}
\newcommand{\da}{$\mathcal{A}^d$\xspace}
\newcommand{\ed}{$\mathcal{ED}$}
%\newcommand{\entity}{e\xspace}
\newcommand{\es}{$\mathcal{E}$\xspace}
\newcommand{\op}{$\theta$\xspace}
%\newcommand{\bnf}{$\varphi$\xspace}
\newcommand{\fobnf}{$\phi$\xspace}
\newcommand{\fobn}{$\mathcal{M}_{FOBN}$\xspace}
\newcommand{\bn}{\textsc{bn}\xspace}
%\newcommand{\fbn}{$\textsc{fobn}$\xspace}
\newcommand{\MV}{$NULL$\xspace}
\newcommand{\mapping}{$f$\xspace}
\newcommand{\stripe}{$S$\xspace}
\newcommand{\selectivity}{\textsl{sel}\xspace}
\newcommand{\size}{\textsl{size}\xspace}
\newcommand{\missing}{\textsl{mratio}\xspace}
\newcommand{\connectivity}{\textsl{cratio}\xspace}
\newcommand{\probDB}{$\mathcal{DB}^p$}
%\newcommand{\exf}{$\mathcal{E}$}

% ICDE10

\newcommand{\tf}{text-string\xspace}
\newcommand{\IPD}{IPS\xspace}
\newcommand{\tfs}{text-strings\xspace}
\newcommand{\Tfs}{Text-strings\xspace}
\newcommand{\ie}{\textsc{ie}\xspace}
\newcommand{\model}{$\mathcal{M}$\xspace}
\newcommand{\tokenset}{\ensuremath{\mathbf{X}}\xspace}
\newcommand{\tokenseq}{\ensuremath{\mathbf{x}}\xspace}
\newcommand{\rvset}{\ensuremath{\mathbf{V}}\xspace}
\newcommand{\token}{\ensuremath{x}\xspace}
\newcommand{\labelset}{\ensuremath{\mathbf{Y}}\xspace}
\newcommand{\labelseq}{\ensuremath{\mathbf{y}}\xspace}
\newcommand{\crf}{$\mathcal{M}_{CRF}$\xspace}
\newcommand{\tokentbl}{\textsc{TokenTbl}\xspace}
\newcommand{\mr}{\textsc{MR}\xspace}
\newcommand{\strs}{$\mathcal{D}$\xspace}
\newcommand{\str}{\textit{d}\xspace}
\newcommand{\probsel}{\textit{probsel}\xspace}
\newcommand{\pwd}{\textit{pwd$(D^p)$}\xspace}

% HFGM-286report

\newcommand{\hfgm}{\textsc{hfgm}\xspace}
\newcommand{\rvs}{\textsc{rv}s\xspace}
\newcommand{\prm}{\textsc{prm}\xspace}

% extensibility

\newcommand{\pgm}{\textsc{PGM}\xspace}
\newcommand{\fctrset}{$\mathcal{F}$\xspace}
\newcommand{\probdbms}{\textsc{PDBMS}\xspace}

\newcommand{\argmax}{\operatornamewithlimits{argmax}}

\usepackage{times}
\usepackage{ams}
\usepackage{amsthm}
\usepackage{algo}
\usepackage{amsmath}
\usepackage{appendix}
\usepackage{graphicx}
\usepackage{xspace}
\usepackage{color}
\usepackage{subfigure}
\usepackage[small,compact]{titlesec}
%\usepackage{ulem} % Strikethrough font.
%\usepackage{abstract}
\usepackage[ruled,vlined]{algorithm2e}
\usepackage{balance}

\newtheorem{example}{EXAMPLE}
\newtheorem{definition}{DEFINITION}
\newcommand{\superscript}[1]{\ensuremath{^{\textrm{#1}}}}


\begin{document}
%
% --- Author Metadata here ---
%\conferenceinfo{SIGMOD}{'13 New York, New York USA}
%\CopyrightYear{2007} % Allows default copyright year (20XX) to be over-ridden - IF NEED BE.
%\crdata{0-12345-67-8/90/01}  % Allows default copyright data (0-89791-88-6/97/05) to be over-ridden - IF NEED BE.
% --- End of Author Metadata ---

\title{{\ttlit CASTLE}: Crowd-Assisted System for\\ Textual
Labeling and Extraction}
%\subtitle{[Extended Abstract]
%\titlenote{A full version of this paper is available as
%\textit{Author's Guide to Preparing ACM SIG Proceedings Using
%\LaTeX$2_\epsilon$\ and BibTeX} at
%\texttt{www.acm.org/eaddress.htm}}}
%
% You need the command \numberofauthors to handle the 'placement
% and alignment' of the authors beneath the title.
%
% For aesthetic reasons, we recommend 'three authors at a time'
% i.e. three 'name/affiliation blocks' be placed beneath the title.
%
% NOTE: You are NOT restricted in how many 'rows' of
% "name/affiliations" may appear. We just ask that you restrict
% the number of 'columns' to three.
%
% Because of the available 'opening page real-estate'
% we ask you to refrain from putting more than six authors
% (two rows with three columns) beneath the article title.
% More than six makes the first-page appear very cluttered indeed.
%
% Use the \alignauthor commands to handle the names
% and affiliations for an 'aesthetic maximum' of six authors.
% Add names, affiliations, addresses for
% the seventh etc. author(s) as the argument for the
% \additionalauthors command.
% These 'additional authors' will be output/set for you
% without further effort on your part as the last section in
% the body of your article BEFORE References or any Appendices.

\numberofauthors{4} %  in this sample file, there are a *total*
% of EIGHT authors. SIX appear on the 'first-page' (for formatting
% reasons) and the remaining two appear in the \additionalauthors section.
%
%\author{
% You can go ahead and credit any number of authors here,
% e.g. one 'row of three' or two rows (consisting of one row of three
% and a second row of one, two or three).
%
% The command \alignauthor (no curly braces needed) should
% precede each author name, affiliation/snail-mail address and
% e-mail address. Additionally, tag each line of
% affiliation/address with \affaddr, and tag the
% e-mail address with \email.
%
\def\ufl{\superscript{*}}
\def\br{\superscript{\dag}}
\def\sharedaffiliation{\end{tabular}\newline\begin{tabular}{c}}
\author{Sean Goldberg\ufl, Jeff Depree\ufl, Daisy Zhe Wang\ufl, and Tim Kraska\br\\
\begin{tabular}{ccc}
\{sean@cise.ufl.edu, jdepree@cise.ufl.edu, daisyw@cise.ufl.edu, kraskat@cs.brown.edu\}\\
\end{tabular}
\sharedaffiliation
\begin{tabular}{ccc}
\affaddr{{\ufl}University of Florida,{\ }} \affaddr{{\br}Brown University{\ }}
\end{tabular}
}
% 1st. author
%\alignauthor{Sean Goldberg\\
%\titlenote{Dr.~Trovato insisted his name be first.}\\
%       \affaddr{University of Florida}\\
       %\affaddr{1932 Wallamaloo Lane}\\
       %\affaddr{Wallamaloo, New Zealand}\\
%       \email{sean@cise.ufl.edu}
%}
% 2nd. author
%\alignauthor{Jeff Depree\\
%\titlenote{The secretary disavows
%any knowledge of this author's actions.}\\
%       \affaddr{University of Florida}\\
       %\affaddr{P.O. Box 1212}\\
       %\affaddr{Dublin, Ohio 43017-6221}\\
%       \email{jdepree@cise.ufl.edu}
%}
% 3rd. author
%\alignauthor Lars Th{\o}rv{\"a}ld\titlenote{This author is the
%one who did all the really hard work.}\\
%       \affaddr{The Th{\o}rv{\"a}ld Group}\\
%       \affaddr{1 Th{\o}rv{\"a}ld Circle}\\
%       \affaddr{Hekla, Iceland}\\
%       \email{larst@affiliation.org}
%\and  % use '\and' if you need 'another row' of author names
% 4th. author
%\alignauthor{Daisy Zhe Wang\\
%       \affaddr{University of Florida}\\
       %\affaddr{Brookhaven National Lab}\\
       %\affaddr{P.O. Box 5000}\\
%       \email{daisyw@cise.ufl.edu}
%}
% 5th. author
%\alignauthor{Tim Kraska\\
%       \affaddr{Brown University}\\
       %\affaddr{Moffett Field}\\
       %\affaddr{California 94035}\\
%       \email{kraskat@cs.brown.edu}
%}
% 6th. author
%\alignauthor Charles Palmer\\
%       \affaddr{Palmer Research Laboratories}\\
%       \affaddr{8600 Datapoint Drive}\\
%       \affaddr{San Antonio, Texas 78229}\\
%       \email{cpalmer@prl.com}
%}
% There's nothing stopping you putting the seventh, eighth, etc.
% author on the opening page (as the 'third row') but we ask,
% for aesthetic reasons that you place these 'additional authors'
% in the \additional authors block, viz.
%\additionalauthors{Additional authors: John Smith (The Th{\o}rv{\"a}ld Group,
%email: {\texttt{jsmith@affiliation.org}}) and Julius P.~Kumquat
%(The Kumquat Consortium, email: {\texttt{jpkumquat@consortium.net}}).}
%\date{30 July 1999}
% Just remember to make sure that the TOTAL number of authors
% is the number that will appear on the first page PLUS the
% number that will appear in the \additionalauthors section.

\maketitle
\begin{abstract}
The need to process and store large amounts of uncertain and imprecise data has necessitated the use of Probabilistic Databases (PDBs) which maintain and allow queries over data that carry a degree of uncertainty.  An integral part of the data cleaning process is finding efficient ways to reduce this uncertainty.  We propose CrowdPillar, a PDB system with an embedded graphical model structure that can employ crowdsourcing techniques such as Amazon Mechanical Turk (AMT) to resolve the most uncertain data entries.  CrowdPillar utilizes functions associated with the entropy over nodes in the graph to select which fields should be submitted to the crowd for correction.  This paper lays out the CrowdPillar system, discusses formulating questions for submission to AMT, and showcases the use of belief theory to combine responses for integration back into the PDB.
\end{abstract}


\section{Introduction}

Motivate need for automatic information extraction \newline

\noindent Discuss reasons for wanting to automatically segment citations \newline

\noindent Introduce system \newline

\noindent Briefly describe dataflow from extraction to selection and integration

% 1.5

\section{Background}

\subsection{Probabilistic Databases}

\subsection{Conditional Random Fields (CRF)}

\subsection{Inference Queries over a CRF Model}

\subsection{Uncertainty Sampling}

\subsection{Crowdsourcing}

% 1.5 (3.5)
% PDBs
% CRFs
% Inference Queries
% crowdsourcing


\section{System Overview}

Graphic showing all system components \newline

\noindent Extraction (CRF) \newline

\noindent Storage \newline

\noindent Selection \newline

\noindent Crowd \newline

\noindent Integration

% 1.5 (5)
% Data Flow
% Data Model (tables)
% Operators

\section{Selection}

Process for selecting a token from each citation \newline

\noindent Clustering of tokens by various metrics (same token neighborhood, same label neighborhood, etc) \newline

\noindent Ranking of clusters by various metrics (high entropy, size of cluster)






% 2.5 (7.5)
% Forward & Backward Tables
% Marginal Inference
% Entropy
% Clustering (Optimization to reduce redunancy)

%\section{HIT Management}

\sean{Discuss various operators for HIT management.}

\sean{Mapping selected citations into questions.}

\sean{Posting questions onto AMT.}

\sean{Checking status, approving results, and retrieving questions.}

\sean{Handling qualification tests.}






% API to post questions
% Sample interface
% Approving and retrieving questions

%\section{Aggregation}

\noindent\sean{Introduction into quality control and using redundancy to improve performance.}


\noindent\sean{Why aren't we content to use MV?}

\subsection{Evaluating Turker Quality}

\noindent\sean{Intermediate table CrowdQuality(workerID,quality) generated from HIT sets using Dawid-Skene}. 

\subsection{Bayesian Aggregation}

\noindent\sean{Prior initialized in evidence table.  Bayesian approach multiplies prior by new evidence and stores conditional back into evidence table.}

\subsection{Entropy Thresholding}

\noindent\sean{Improving trustworthiness of evidence by removing values above a certain entropy threshold.} 

% Quality Table
% Crowd Aggregation/Evidence Table

\section{Integration}

\sean{Add paragraph on "frequentist" (MV) versus "Bayesian" approach}
One of the difficulties in relying on information from a crowd of sources is the possibility of a high degree of noise due to unreliable and in some cases even malicious sources.  One of the standard procedures for increasing quality control is to increase the redundancy of questions.  By asking the same question to multiple sources and aggregating the answers, we can achieve a higher probability of a good answer.

In many cases, it suffices to collect, say, 3 or 5 votes on each question and use the majority opinion.  There are potential scenarios in which this ceases to be an effective strategy.  If the probability of receiving low quality work is equal to or greater than that of receiving higher quality, it's detrimental to treat every vote of equal merit.  Confusing or difficult questions can also cause conflict among the workers and result in a mix of answers.  Taking the deterministic mode results in a loss of information about the controversy of the question, information which may prove useful in applications such as sentiment analysis or opinion-dominated questions.

Thus we are led to a desire to manifest the crowd response probabilistically, weighing votes proportionately and making decisions when conflicted on a question.  We implement two approaches for this data integration task, drawing separately from probability theory and belief theory.  The first maintains a single probability function, establishing a prior based on the machine's labeling, and updating the posterior using Bayes's Rule.  Alternatively, we combine the Turker response in the absence of the machine prior using Dempster-Shafer theory.  Both methods require an identification of the level of quality of each individual Turker.  We describe previous work that we've leveraged in the next section before outlining our two integration methods.\sean{Still may want to add Halpern and Fagin reference.}

\subsection{Evaluating Turker Quality}

Amazon Mechanical Turk provides no working system for maintaining the quality and reliability of their workforce and it is generally up to the Requester to ascertain such values on their own.  The simplest system, known as "honey potting", is to carefully intermix questions for which the answer is known in advance and judge Turker performance against the gold standard.  While generally effective, it lacks robustness and is defeatable to smart enough Turkers that can recognize them over time.  More sophisticated methods estimate quality an unsupervised manner by judging each Turker's level of agreement with the mean set of answers.  Examples include Bayesian \cite{citeulike:9437699, DBLP:journals/jmlr/RaykarY12} methods and an approach using majority vote and expectation maximization \cite{Ipeirotis:2010:QMA:1837885.1837906}.

We focus on a modified version of latter, attributable to Dawid and Skene \cite{1979}, for implementation into \sysName .  For each question the EM algorithm takes a set of answers $a_{1}$,...,$a_{N}$ provided by N Turkers assumed to be drawn from a categorical distribution.  Associated with each Turker is a latent "confusion matrix" $\pi^{k}_{ij}$ that designates the probability the $k^{th}$ Turker will provide label $j$ when true answer is $i$.  Our modification simplifies to a binary accuracy variable $\pi^(k)$, which represents probability they will correctly label a question with the true answer.  The goal of Dawid and Skene's EM algorithm is to recover $\pi^{k}$ in the presence of the answers $a^{m}_{1}$,...,$a^{m}_{k}$ for a set of questions $m \in M$.

In order to obtain a sufficient number of answers to similar questions by, HITs are designed in higher cost blocks.  The single task of supplying a label to a token is worth around \$0.01.  HITs are packaged in groups of 10 questions at \$0.10 each.  This ensures that if $K$ Turkers answer the HIT, relative performance can be judged across all 10 questions.

The algorithm initializes each Turker's accuracy to 1.  It takes a majority vote among the answers to each question to define an initial answer set.  Based on this agreed upon answer set, each Turker's accuracy $\pi^{k}$ is computed.  Another majority vote weighted by $\pi^{k}$ determines a possibly different answer set.  The Turker accuracies are re-computed.  This process continues until convergence in both the "true" answer set and the $\pi^{k}$ accuracies.

Let us take a moment to define precisely how we interpret Turker quality in the context of results of the EM algorithm.  While the Dawid \& Skene approach ultimately is calculated as correct or incorrect accuracies from a set of questions, we assume a different characteristic behavior associated with this score.  Instead of the quality being a measure of whether we believe the Turker is "correct" or not, we take quality to be a measure of \textit{reliability}.  The quality score models the probability the Turker knows the correct answer and selects accordingly, while the inverse is the probability of a \textit{random guess} from the set of possible answers.  The two approaches in the next section tackle the problem of combining responses once we have an estimate of the Turkers' quality or reliability. 

\subsection{Two Approaches to Integration}
%~ \eat{
%~ The reason for submitting to belief theory as our main tool in the aggregation of the Turkers and machine is that it provides a natural framework arriving at a posterior distribution composed of various pieces of evidence.  While the roots of belief theory first centered around the Dempster-Shafer model, much criticism has been laid upon the model for turning up erroneous or inaccurate results.  Halpern and Fagin \cite{DBLP:journals/ai/HalpernF92} argue this is purely from a misuse of appropriating one interpretation for another.  The first view of belief function one can take is that of a generalized probability function, starting with a prior probability and updating as new evidence comes along to arrive at a conditional posterior.  On the other hand, viewing belief functions as evidence themselves leads one to use Dempster's Rule of Combination.  One presents the \textit{updating} of evidence while they other presents the \textit{combining}.  One utilizes a prior while the other does not.
%~ 
%~ We use this as inspiration for studying two different approaches to aggregating humans and machines akin to the differing interpretations.  In our Bayesian formulation, the CRF marginal distribution is used as a prior and \textit{updated} based on Turker responses.  Using an alternative Dempster-Shafer model, we forego the use of a prior and \textit{combine} Turker responses using Dempster's Rule of Combination.  
%}



\subsubsection{Bayesian Conditional Probability}

The fundamental assumption taken with the Bayesian model is that the ML extracted values present a serviceable prior probability over the choice of labels.  For a well-trained machine model, its output can be used as starting point upon which additional evidence from the crowd is used to adjust the label decision in the right direction. The machine acts as a regularizer, the more peaked any aspect of the original output distribution the more impact the prior plays and consequently the greater the trust placed in the original model.

Let $A^{n}_{1}$,...,$A^{n}_{K}$ be a set of categorical random variables corresponding to the answers received from $K$ Turkers for question $n$.  The CRF's original output, a random variable $L$ which also follows a categorical distribution over the label space, is our current estimate of the true distribution of labels fora specific token.  The integration problem is to find the posterior $P(L^{n}|A^{n}_{1}$,...,$A^{n}_{K})$ conditioned on the answers provided by the Turkers.  This can be calculated using Bayes's Rule:     

\begin{equation}
P(L^{n}|A^{n}_{1},...,A^{n}_{K}) = \frac{P(A^{n}_{1},...,A^{n}_{K}|L^{n})P(L^{n})}{P(A)}
\end{equation}

Since the set of answers is fixed and we're only concerned with relative differences among different label possibilities, we may without loss of generality focus solely on the numerator.  The initial prior, $P(L)$, is just the CRF's marginal probability before considering any new evidence.  The evidence term, $P(A^{n}_{1},...,A^{n}_{K}|L^{n})$, represents the probability the Turker answers were generated from a specific true label.  Our Bayesian model assumes Turker quality is an adequate measure of their agreement with the true label,

\begin{equation}
\label{eq:independence}
P(A^{n}_{1},...,A^{n}_{K}|L^{n}) = \prod_{k}P(A^{n}_{k}|L^{n})
\end{equation}

\begin{equation}
\label{eq:bayes_evidence}
%P(A^{n}_{k}=a|L^{n}=l) = |\mathbbm{1}_{{a}\neq l} - Q_{k}| + |\mathbbm{1}_{a=l}-Q_{k}|*\frac{1}{|L|}
P(A^{n}_{k}=a|L^{n}=l) = 1{\hskip -2.5 pt}\hbox{I}_{{a}= l}* Q_{k} + (1-Q_{k})*\frac{1}{|L|}
\end{equation}

where $a$ and $l$ are values drawn from the label space and $Q_{k}$ is the quality of the k$^{th}$ worker.  Equation~\ref{eq:independence} follows from all Turker answers being independent of each other and equation~\ref{eq:bayes_evidence} simply restates our assumption about the use of Turkery quality $Q_{k}$.  If the answer matches the label $l$, the first term on the right hand side is the probability the Turker is reliable and answers the question truthfully.  The second term incorporates the probability they are unreliable or a spammer and through \textit{random guessing} finds the correct answer with probability $1/|L|$, $|L|$ being the number of possible labels.  If they don't match, we have the probability the Turker is unreliable, $1-Q$, and the probability a random guess produces an incorrect answer, $(L-1)/L$.
 
The full model is

\begin{align}
\label{eq:full_bayes}
P(L^{n}=l&|A^{n}_{1}=a_{1},...,A^{n}_{K}=a_{k}) = \nonumber\\
                 &P(L^{n}=l)\prod_{k}\big(1{\hskip -2.5 pt}\hbox{I}_{{a_{k}}=l} *Q_{k}  + (1-Q_{k})*\frac{1}{|L|}\big)
%P(&L^{n}=l|A^{n}_{1}=a_{1},...,A^{n}_{K}=a_{k}) = \\
%&P(L^{n}=l)\prod_{k}\big(|\mathbbm{1}_{{a_{k}}\neq l} - Q_{k}|  + |\mathbbm{1}_{a_{k}=l}-Q_{k}|*\frac{1}{|L|}\big)
\end{align}

Using equation~\ref{eq:full_bayes} for all possible labels $l$ and renormalizing produces a new posterior distribution accounting for both the initial ML extracted result and evidence gathered from the crowd.  The product can be extended and updated as new evidence comes in over time.  While currently evidence is designed to come from the crowd in \sysName , there is no explicit restriction preventing future updates from incorporating evidence from a number of different extractions as well as the crowd.  We conclude this section with an explicit example.

\sean{Probably want to change these numbers.  What differences between DS and Bayes do I want to exhibit by using specific numbers?}
EXAMPLE 1. \textit{
Assume a binary question is answered by two Turkers.  Turker A has quality 0.8 and answers label 0, while Turker B has quality 0.6 and answers label 1.  The prior CRF marginal probability over \{0,1\} is \{0.3,0.7\}.  We want to ascertain the combined distribution for the label $L$.  According to equation~\ref{eq:full_bayes}, 
%\begin{equation}
\begin{align}
P(L=0|A,B) &= \frac{1}{Z}P(L=0)P(L=0|A)P(L=0|B)\nonumber\\
	        &= (0.4)*(0.9)*(0.2)*\frac{1}{Z}\nonumber\\
	        &= .072*\frac{1}{Z}\\
P(L=1|A,B) &= \frac{1}{Z}P(L=1)P(L=1|A)P(L=1|B)\nonumber\\
	        &= (0.6)*(0.1)*(0.8)*\frac{1}{Z}\nonumber\\
	        &= .048*\frac{1}{Z}
\end{align}
%\end{equation}
After combining and normalizing, the final distribution over \{0,1\} is \{0.6,0.4\}.  While the CRF originally favored label 1, the new distribution favors label 0.
}

\subsubsection{Dempster-Shafer Evidential Combination}
%~ \eat{
%~ A viable alternative is to exhibit no faith in the machine's initial marginal calculation.  After all, one could argue that by selecting only the most uncertain tokens that metric loses its value.  
%~ }

Without explicit reference to the CRF prior, we're left with the task of \textit{combining} disparate evidence from a group of Turkers.  This can be accomplished using Dempster's Rule of Combination, which operates over a set of mass functions.  Mass functions differ from probability functions by relaxing the Kolmogorov axiom that functions must sum to 1.

While the Bayesian approach was inspired by an alternative interpretation of belief functions, the actual implementation is still a probability function through and through, with all of Kolmogorov's axioms defining a probability function still holding.  For evidential combination, however, we leverage the full power of belief theory and relax some of those axioms to map to a set of belief functions.  

~ The main difference between a belief function and a probability function is that probability functions are defined only over the \textit{measurable} subsets of a set while belief functions are defined over \textit{all} subsets (the power set) of a set \cite{shafer1976mathematical}.

We now describe mapping of the Turker data these mass functions.  Like with the Bayesian approach, our confidence in them getting the answer correct is reflected in their Quality score.  The mass function $m(a_{k})$ gets assigned the score $Q_{k}$.  Let $\mathcal{A}$ be the set of all possible labels ${1,2,...,L}$.  Intuitively, $m(\mathcal{A})$ is the mass associated with a random guess and all $L$ labels being equally likely.  We assume in this framework that Turkers are either reliable, getting the answer correct with belief score $Q_{k}$, or unreliable, reflected in a random guess with belief score $1-Q_{k}$.  Explicitly, for a Turker $k$ with provided answer $a_{k}$:

\begin{equation}
m^{n}(2^{L}) = 0
\end{equation}
\begin{equation}
\label{eq:mass1}
m^{n}(a_{k}) = Q_{k}
\end{equation}
\begin{equation}
\label{eq:mass2}
m^{n}(\mathcal{A}) = 1-Q_{k}
\end{equation}

The first equation simply states that initialize all mass functions to zero before setting the two values below.  The mass function $m(\mathcal{A})$ has no meaning in standard probability theory, as the set of all outcomes is not a measurable in the probabilistic sense.  We use it mainly as bookkeeping for the uncertainty in the result before normalizing it out when the aggregation computation is completed.  The set of mass functions from multiple Turkers can be combined using Dempster's Rule of Combination between Turker 1 and Turker 2 for each set $A\in2^{L}$:

\begin{equation}
\begin{split}
\label{eq:DS_combo}
m_{0,1}(A) &=(m_{1}\oplus m_{2})(A)\\
                   &=\frac{1}{1-K} \sum_{B\cap C=A\neq\emptyset} m_{1}(B)m_{2}(C)
\end{split}
\end{equation}

\begin{equation}
K=\sum_{B\bigcap C=\emptyset}m_{1}(B)m_{2}(C)
\end{equation}

The procedure is to map all HIT responses to mass functions and combine them one-by-one in turn to produce a single combined mass function.  Any remaining uncertainty in $m_{comb}(\mathcal{A})$ is added to all the singleton functions and re-normalized to produce a single probability function.  The original belief formulation is maintained in \sysName for easy combination if new evidence arrives at a later time.

\textit{
EXAMPLE 2. Given the same Turkers and answers from EXAMPLE 1, Dempster's Rule of Combination may also be used to combine them.  First, we map them to mass functions using equations~\ref{eq:mass1} and ~\ref{eq:mass2}:
\begin{align}
m_{A}(0) = 0.8,  m_{A}(1) = 0,  m_{A}(0,1) = 0.2\nonumber\\
m_{B}(0) = 0,  m_{B}(1) = 0.6,  m_{B}(0,1) = 0.4
\end{align}
We apply equation~\ref{eq:DS_combo} to combine Turkers A and B.
\begin{align}
m_{A,B}(0) = m_{A}(0)*m_{B}(0,1) = 0.32\nonumber\\
m_{A,B}(1) = m_{B}(1)*m_{A}(0,1) = 0.12\nonumber\\
m_{A,B}(0,1) = m_{A}(0,1)*m_{B}(0,1) = 0.08
\end{align}
To convert to a probability distribution, we add $m_{A,B}(0,1)$ to each of the individual components and normalize.  The final distribution over \{0,1\} is \{0.66,0.34\}.  The contrasts with the result of EXAMPLE 1 by the exclusion of a machine prior.
}

While we introduce Dempster-Shafer theory here in the context of our simpler one-answer-per-question framework currently found in \sysName , it is not to be taken in contrast with its Bayesian counterpart, but as a generalization of it.  The method will become more powerful in future work when we plan to extend functionality to allow the Turkers to provide more than one response per question when uncertain.  Reasoning over such fuzzy sets exemplifies the real power for using belief theory over probability theory.

\subsection{Quantifying Turker Performance}

Even human computation is not perfect.  The previous section looked at ways to combine Turker answers probabilistically to arrive at a final result that is not deterministic.  This is useful for when there is controversy or confusion elicited over the answers of a question.  We use the entropy of the final label distribution to arrive at confidence value for each question.  Depending on the required accuracy of the application, a threshold on the confidence may be placed to assure only the highest quality results make it through.  In the experiments we highlight Receiver Operating Characteristic (ROC) curves that measure performance vs. answer recall.  Answers not making the cut may have their questions re-submitted to attain more information in discerning the result. 

\subsection{Probabilistic Integration}
\label{sec:probInt}

\begin{algorithm}[fillcomment]
\label{alg:integration}
\SetKwInOut{Input}{input}\SetKwInOut{Output}{output}
\Input{Array of turker answers A,\\
ML prior label dist. M,\\
CRF model C,\\
Unlabeled document d,\\
Token t}
\Output{Labeled document d\_labeled}
\BlankLine
\CommentSty{//Estimate Turker qualities from answers}\;
Q = Dawid\_Skene(A)\;
\CommentSty{//Compute posterior distribution of answers}\;
\If{Bayesian}{
	combination = Bayesian\_integrate(M,A,Q)\;
}
\If{Dempster-Shafer}{
	combination = DS\_integrate(A,Q)\;
}
\CommentSty{//Integrate back into model}\;
label = index\_of\_max(combination)\;
d\_labeled = constrained\_Viterbi(C,d,t,label)\;

\caption{Probabilistic integration through constrained Viterbi.}
\end{algorithm}

Not only does \sysName have the ability to aggregate answers from multiple sources, but also the ability to reinsert the resulting distribution back into the CRF. Since the underlying architecture of the system is a CRF, the dependence properties of each field are made explicit and re-running the inference algorithm has the potential to change surrounding fields as well.  This constrained inference substitutes the aggregated marginal distribution of a token in for the computed transition probabilities and highlights a very strong advantage of \sysName system in that large errors can be corrected by small, incremental changes.

Algorithm~\ref{alg:integration} shows the basic outline of our probabilistic integration scheme.  Turker answers pulled from AMT are a set of labels for token $t$ from document $d$.  They're used in the Dawid \& Skene EM method to estimate the Turkers' individual quality measures.  Depending on the integration technique chosen, either Bayesian or Dempster-Shafer is used to compute the final posterior distribution over $t$'s possible labels.  The max likelihood label is used in the constrained Viterbi function over $d$.  This function computes Viterbi in a similar manner as the original, except that all paths not passing through the max likelihood label for $t$ are set to zero.


% Constrained Inference

\section{Experiments}
In this section we demonstrate the effectiveness of our selection and integration approaches on sets of both synthetic and real data.  We extracted 14,000 labeled citations from DBLP \sean{footnote 1} and 500,000 from the PubMed database \sean{footnote 2}.  For unlabeled testing data, we removed the labels and concatenated text from each of the available fields.  Order of fields was occasionally mixed in keeping with real-life inconsistency of citation structure.

\subsection{Experiments w/ Synthetic Data}
\subsubsection{Selection}

\begin{figure*}[t]
	\centering
	\subfigure[High Entropy] {
		\includegraphics[width=0.48\textwidth]{images/selection_exp1_highE.png}
		\label{fig:first1}
	}
	\subfigure[Total Entropy] {
		\includegraphics[width=0.48\textwidth]{images/selection_exp1_totalE.png}
		\label{fig:second1}
	}
	\caption{Seeding comparison for high entropy and total entropy ranking.}
	\label{fig:select1}
\end{figure*}

\begin{figure*}
	\centering
	\subfigure[High Entropy] {
		\includegraphics[width=0.48\textwidth]{images/selection_exp2_highE.png}
		\label{fig:firs2t}
	}
	\subfigure[Total Entropy] {
		\includegraphics[width=0.48\textwidth]{images/selection_exp2_totalE.png}
		\label{fig:second2}
	}
	\caption{Clustering comparison for high entropy and total entropy ranking.}
	\label{fig:select2}
\end{figure*}

\begin{figure*}
	\centering
	\subfigure[High Entropy] {
		\includegraphics[width=0.48\textwidth]{images/selection_exp3_highE.png}
		\label{fig:first3}
	}
	\subfigure[Random] {
		\includegraphics[width=0.48\textwidth]{images/selection_exp3_random.png}
		\label{fig:second3}
	}
	\caption{Ranking comparison for high entropy and total entropy ranking.}
	\label{fig:select3}
\end{figure*}

Figures~\ref{fig:select1},~\ref{fig:select2}, and~\ref{fig:select3} contain experiments comparing our various selection algorithms by detailing the accuracy improvements for each question asked.  Tokens were selected using a specific combination of seeding, clustering, and ranking approaches.  

Initially, a token was selected from each document using some seeding mechanism.  The number of questions is prohibitively large to show the full range of our methods, so we automatically answer each question with its ground truth label.  It's shown in the next section that the high accuracy of Mechanical Turk answers allow this to be a working assumption.  The same answer (label) to the question (token) is applied to all subsequent tokens in its cluster.  A constrained Viterbi inference algorithm runs over all documents  containing tokens belonging to question clusters.  The accuracy value in each figure represents the final token accuracy after running constrained inference.

In this paper, we proposed two possible functions for selecting a token from each document.  High Entropy chooses that which has the highest marginal entropy over its labels while Neighborhood Entropy selects the token in the center of the largest 3-window pocket of marginal entropies.  Figure~\ref{fig:select1} shows effectiveness of both methods when compared to randomly selecting a token for both High Entropy and Total Entropy ranking.  The default clustering is Same Label Neighborhood.  In both cases, Neighborhood Entropy maintains a consistently higher accuracy, lending evidence to the idea that constrained inference has a larger effect on pockets of high entropy than it does on the single highest entropy tokens.  Both methods double the overall possible accuracy improvement with fewer questions.  For some accuracy regions even orders of magnitude fewer questions are needed.

Figure~\ref{fig:select2} compares the possible clustering algorithms for the High Entropy and Total Entropy ranking functions.  All use high entropy for seeding.  Clustering by similar tokens that have the same label and share preceding and succeeding labels produce the largest clusters with the greatest net effect.  For the DBLP set, there were zero clustering errors for Same Token and Same Field, and approximately 2\% of citations were clustered incorrectly using the Same Label approach.  As the figures prove, however, the benefit of larger clusters far outweigh the additional errors.

The final set of synthetic selection experiments is shown in Figure~/ref{fig:selection3}.  While it initially seemed like a heuristic, the effectiveness of Total entropy for ranking should now be apparent.  For both high entropy and random seeding, total entropy combines the early question strength of large clusters and the late question power of high entropy.  Same Label Neighborhood is again the default clustering for all ranking comparisons.  It's important to note, that even for random seeding, Total Entropy outperforms everything else.



\subsubsection{Integration}
\begin{figure}
		\includegraphics[width=0.48\textwidth]{images/integration_exp1_numT.png}
		\label{fig:integrate1}
		\caption{Comparison of integration methods vs. number of Turkers per question.} 
\end{figure}
\begin{figure}
		\includegraphics[width=0.48\textwidth]{images/integration_exp2_meanQ.png}
		\label{fig:integrate2}
		\caption{Comparison of integration methods vs. average Turker quality.} 
\end{figure}

Answers received from the crowd have many variables that must be factored into a rigorous justification of any method of combination.  Primarily, we are concerned with measuring how the final combined accuracy is affected both by the number of redundant answers and by the actual quality of the workers.  A set of synthetic responses to real questions were generated in a manner that the allowed the average worker quality to be varied throughout the experiments.

Workers were automatically generated by selecting a quality value $Q \in [0,1]$ from a Gaussian distribution of standard deviation 0.3 and a mean that varies over the experiment.  Quality values drawn outside the $[0,1]$ range were truncated at the boundary.  Each worker was assigned to a 'HIT', which constituted a set of 10 questions.  The quality level dictated the generation of answers.  In keeping with our assumption of quality, the true label was applied with probability $Q$.  With probability $1-Q$, the answer was drawn from a uniform distribution over the label space.  In this manner we assembled 500 questions answered by 3-13 workers each, with new sets of workers generated every 10 questions.

Figure~\ref{fig:integrate1} shows how the integration methods outlined in this paper compare as we increase the redunancy of question asking.  The mean worker accuracy is 0.5 in these results and the prior used in the Bayesian method is uniform.  While all methods increase monotonically as expected, the Bayesian method produces the best results for low redundancy and both Bayesian and Dempster-Shafer are able to attain 100\% accuracy by 13 workers, whereas Majority Voting is not.  The Bayesian method is able to beat Dempster-Shafer slightly due to its uniform prior assumption, which closely follows the distribution of labels for low quality workers.

The availability of high or low quality workers is certain to affect the comparisons, so in Figure~\ref{fig:integrate2} we compare the accuracy of answers as we vary the quality.  Variation is achieved by shifting the center of the Gaussian which produces worker quality values.  We initially set it at 0.2 and shifted to a maximum of 0.8.  As before, the Bayesian and Dempster-Shafer approaches show that managing Turker quality, even when so low as to be just slightly better than random, produces large gains in accuracy.  One method of attaining only high quality workers on Amazon Mechanical Turk is to implement a qualification test that workers must pass before they can complete your tasks.  Our experience has shown that while this does lead to more abled workers, the price paid in time can be many times slower.  The results of Figures~\ref{fig:integrate1} and~\ref{fig:integrate2} show that with a better integration method, some of the constraints designed to achieve higher quality may be relaxed without a large decrease in accuracy, key to making \sysName fast, agile, and powerful.

\sean{Exp6: Recall vs. accuracy for varying entropy thresholds.}

\subsection{Experiments w/ Real Data}
\sean{Description of real experiment methodology.}

\sean{Exp7: Table of accuracy comparisons for DS, MV, and Bayes before and after edits plus clamped inference for both data sets.}

\sean{Exp8: Recall vs. accuracy for varying entropy thresholds for both data sets.}

% 1.5 (10.5)

\section{Related Work}
\sean{Reference further attempts at using CRF for IE.}
Implementations of probabilistic databases include TRIO \cite{DBLP:conf/vldb/AgrawalBSHNSW06}, MystiQ \cite{Boulos:2005:MSF:1066157.1066277}, MayBMS \cite{Huang09maybms:a}, and Orion \cite{DBLP:conf/comad/SinghMMPHS08}.
\sean{Might switch related work and background or move related work to back}
Discuss previous attempts at citation IE (esp. by McCallum) \newline

\noindent Identify various methods of data cleaning/integration \newline

\noindent Compare selection process to that used in active learning \newline

\noindent Introduce crowdsourcing and related quality control efforts

% .5 (2)

\section{Conclusion}
\balance

% .25

\section{Future Work}

% .25 (11)

% References (12)
\bibliographystyle{abbrv}
\bibliography{sigproc}  % sigproc.bib is the name of the Bibliography in this case
%

%\section{Appendix}

\end{document}
